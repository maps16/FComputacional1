\documentclass[12pt]{article}
\usepackage[spanish,mexico]{babel}
	\selectlanguage{spanish}
\usepackage{graphicx}
\usepackage{amsmath}
\usepackage{wrapfig}
\usepackage{float}
\usepackage[utf8]{inputenc}

\usepackage{graphicx}
\graphicspath{{images/}}

%\usepackage{vmargin}
%\setmarginsrb{3 cm}{2.5 cm}{3 cm}{2.5 cm}{1 cm}{1.5 cm}{1 cm}{1.5 cm}

\title{Actividad 2: Elementos de la programación Python 1}
\author{Martin Alejandro Paredes Sosa}
\date{Febrero 2016}

\begin{document}

\maketitle

\section{Introducción}
Esta practica consistio en la realización de los primeros programas en Python del curso de \textit{Física Computacional}. Python es un lenguaje de programación interpretado cuya filosofia se enfoca en la sintaxis que favorezca el codigo legible. Es un lenguaje multiparadigma, ya que soporta lenguaje orientado a objetos, programación imperativa y en menor medida programación funcional. \cite{PyWiki}


\section{Problemas Realizados}

Como ya se menciono antes, esta actividad consistio en la realización de los primeros programas en Python, lo cuales nos permitieron tener un mejor entendimiente de como funciona el lenguaje.\\

Se realizaron un total de 5 problemas, en los cuales se utilizaba diferentes herramientas de Python y sus librerias.

\subsection{Problema 1}

``Se deja caer una pelota desde el techo de una torre de altura h. Se desea saber la altura de la pelota respecto a la torre a un determinado tiempo después de haber sido dejada caer."\cite{act}

\subparagraph*{Programa original}

\begin{verbatim}
h = float(input("Proporciona la altura de la torre: "))
t = float(input("Ingresa el tiempo: "))
s = 0.5*9.81*t**2
print("La altura de la pelota es", h-s, "metros")
\end{verbatim}

En base a este codigo, se modifico para que se introduzca la altura($m$) de la torre y se obtenga un tiempo de vuelo.

\subparagraph*{Programa obtenido: caida.py}
\begin{verbatim}
#Importar funcion sqtr
from math import sqrt

#Pedir al altura a usuario
h = float(input("Proporciona la altura de la torre: "))

g=9.81 #Constate de la gravedad

t=sqrt(2*h/g) #Calculo de tiempo de caida

#Impresion de Resultados
print'El tiempo de caida para una altura ',h,'m es: ', t 

\end{verbatim}

\subsection{Problema 2}

`` Un satélite orbita la Tierra a una altura h, con un periodo T en segundos.
Demuestre que la altitud h del satélite sobre la superficie de la Tierra esta dado por la expresión:

\begin{equation} \label{1}
(R+h)^3=\frac{GMT^2}{4\pi^2}
\end{equation}

donde $G = 6.67\times10^{-11} \frac{Nm^2}{kg^2} $ es la constante de Gravitación Universal de Newton, $M = 5.97 \times 10^{24} kg$ es la masa de la Tierra y $R=6371 km$ es su radio.

\subparagraph*{Demostración}
Suponiendo una órbita circular, tendremos un satélite que orbite a velocidad constante, aplicando la Segunda Ley de Newton, considerando que la aceleración sería centrípeta:

\begin{equation} \label{2}
F=m\frac{v^2}{r}
\end{equation}

Siendo $F$ la fuerza gravitatoria entre la Tierra y el satélite:

\begin{equation} \label{3}
F=\frac{GMm}{r^2}
\end{equation}

Expresamos a la velocidad en términos del período del satélite:

\begin{equation} \label{4}
v=\frac{2\pi r}{T}
\end{equation}

Sustituyendo $v$ en \eqref{2} e igualando con \eqref{3} obtenemos:

\begin{equation} \label{5}
\frac{GMm}{r^2}=m\frac{{(\frac{2\pi r}{T})}^2}{r}
\end{equation}

En el caso más general de un satélite orbitando un planeta, la altura de su órbita no es despreciable respecto al radio de la órbita, por lo que la distancia $r$ es igual a la suma del radio del planeta $R$ más la altura $h$ de la órbita sobre su superficie. Por lo que $r=(R+h)$. Simplificando \eqref{5} y sustituyendo a $r$, llegamos a \eqref{1}:

\begin{equation}
(R+h)^3=\frac{GMT^2}{4\pi^2}
\end{equation}







\subsection{Problema 3}
\subsection{Problema 4}
\subsection{Problema 5}
\begin{thebibliography}{6}

	\bibitem{PyWiki}
	Wikipedia,(2016)
	\emph{Python}. Recuperado de\\ https://es.wikipedia.org/wiki/Python

	\bibitem{act}
	Lizárraga, C. (2016)
	\emph{Actividad 2 (2016-1)}. Recuperado de\\ http://computacional1.pbworks.com/w/page/104476954/Actividad\%202\%20(2016-1)
\end{thebibliography}

\end{document}
